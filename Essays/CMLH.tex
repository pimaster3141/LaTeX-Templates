\documentclass[12pt]{article}
\usepackage{fancyhdr}     % Enhanced control over headers and footers 
\usepackage[T1]{fontenc}  % Font encoding
\usepackage{mathptmx}     % Choose Times font 
\usepackage{microtype}    % Improves line breaks      
\usepackage{setspace}     % Makes the document look like horse manure 
%\usepackage{lipsum}       % For dummy text
\usepackage{graphicx}
\usepackage[
top    = 1in,
bottom = 1in,
left   = 1in,
right  = 1in]{geometry}
\newcommand{\myName}{Jason Yang}


\pagestyle{fancy} % Default page style 
\lhead{\myName}
\chead{\hspace{1cm}\small\textit{HW \#10}}
\rhead{\thepage}
\lfoot{}
\cfoot{}
\rfoot{}
\renewcommand{\headrulewidth}{.1pt}
\renewcommand{\footrulewidth}{.1pt}

\thispagestyle{empty} %First page style 

\begin{document}
	\begin{center}
		%\begin{tabular}{c}
		\textbf{\large\myName} \\
		\textbf{\today} \\
		\textit{\small HW \#10}
		%\end{tabular}
	\end{center}

\section{Suppression Rates of Epileptic Seizures and VNS Operating Frequencies}
	Though many forms of epilepsy can be mitigated or completely suppressed using neurological medications, certain forms of epilepsy do not respond well, or are exceedingly drug resistant. In such drug resistant cases, one potential course of action is to employ the use of a vagal nerve stimulator (VNS). a VNS is a type of brain stimulator where the electrodes are wrapped around the left vagus nerve at the carotid sheath. The stimulator module, like a typical deep brain stimulator, is placed underneath the clavical with electrode leads routed through the neck. Figure \ref{fig:vns} shows the location of the stimulator module and electrode placement. \\
	
	\begin{figure}[htbp]
		\centering
		%\includegraphics[width=0.5\linewidth]{vns.jpg}
		\caption{Vagus Nerve Stimulator (VNS): electrode placement and pulse generator location.}
		\label{fig:vns}
	\end{figure}
	
	
	The stimulator generates a current based pulse waveform with an amplitude of about 1.0 to 3.0mA, pulse width from 130 to 500 $\mu$s and a stimulation frequency from 20 to 30Hz. Specific settings can be tuned for the individual's response to the therapy using a magnetic probe. \\
	
	Though the specific method which VNS devices inhibit epileptic seizures is not well understood, the procedure has shown promising results. Several long term studies have illustrated >50\% reduction in seizures in at least 35\% of patients between 1988 and 1995. Furthermore, 20\% of the study population exhibited significant reduction (>75\%) in epileptic events after at least one year. 
	
\newpage
\section{If a feedback loop were incorporated into to VNS. What brain waves would be monitored and at what sampling frequencies? Justify your sensor design parameter}

	One potential solution to treating epilepsy, and a topic of many research focuses, is the idea of being able to predict or detect the onset of a seizure and triggering stimulation (in this case a VNS) to counter it. This idea; however, requires the acquisition of long term EEG recordings, which is a challenge in itself and will not be discussed today. Implementation details aside, once the EEG is acquired, the second challenge is predicting the onset of the seizure. \\
	
	It has been shown that clinical seizure onset is predicated by electrical seizure onset by about 7.5s. This electrical prediction is based upon the spectral information of the EEG signal. One study has shown that prior to seizure onset, the largest fundamental frequencies in EEG lies from 3-8Hz (Theta) and 15-20Hz (Beta) as depicted in Figure \ref{fig:eeg}; however, recent research has shown correlation of seizure with EEG information in the 250Hz band. \\
	
	\begin{figure}[htbp]
		\centering
		%\includegraphics[width=0.5\linewidth]{eeg.png}
		\caption{Histogram of Highest fundamental frequencies at seizure onset.}
		\label{fig:eeg}
	\end{figure}

	In order to accurately capture information in these bands, at minimum, the Nyquist criterion must be upheld which states that the maximal spectral information that can be extracted from a waveform is one half of the sampling frequency: $f_h \le \frac{f_s}{2}$. Good engineering practice; however, recommends at least 10 times the maximal spectral information in order to ensure proper reconstruction of the signal, free from noise artifacts. Furthermore, oversampling will allow for better attenuation of the necessary anti-aliasing filter in order to ensure no artifacts are reflected in the captured waveform. From these requirements, it is recommended that the sampling rate be at least 200Hz for data capture up to Beta waves, and at least 2.5kHz should information in the 250Hz band be desired, with anti-aliasing filters with -3dB cutoffs at 25Hz and 300Hz for each, respectively. \\
	
	It should be noted; however, that there exists a tradeoff between power consumption, sampling rate and resolution. For battery powered devices, it may be prohibitively inefficient to sample at 2.5kHz, thus analysis of the 250Hz band should be restricted to short term or non ambulatory EEG recorders. 
	
	It should be further noted that frequency domain analyses are typically insufficient to accurately predict seizures for general populations. Research has shown that frequency analysis combined with patient-specific machine learning has yielded significantly improved sensitivity to epileptic seizure onset. 
\end{document}

\setlength\headheight{15pt} %Slight increase to header size